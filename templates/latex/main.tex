% This file has example content to demo the template settings

% ============================================
\section{Lorem text}

\lipsum[1-2]

% ============================================
\section{Typeography and fonts}

These packages provide many symbols, fonts, and commands.
The examples below are a small subset which I've used in the past.
% TODO (2024.01.06): Make this a nice table
\begin{itemize}
    \item \texttt{amsmath} provides many useful commands like \texttt{text}
    \item \texttt{amssymb} provides symbols like
        \texttt{triangleq}: $\triangleq$
        and fonts like \texttt{mathbb}: $\QQ$
    \item \texttt{mathrsfs} provides \texttt{mathscr} font: $\mathscr{H}$
    \item \texttt{bbm} provides \texttt{mathbbm} font: $\mathbbm{1}$
\end{itemize}


% ============================================
\section{Math (\(\lambda\))}
\label{sec:SectionToReference}

Standard form of a hyperplane:

\begin{equation}
    \label{eq:MyEquation}
    \boldsymbol{\theta}^T \mathbf{x} + \theta_0 = 0
\end{equation}


% ============================================
\section{Tables and figures}

\subsection{Standard \texttt{tabular}}

\begin{center}
\begin{tabular}{|>{\bfseries}l|l|l|l|l|}
    \hline
    Tenor $T_i$ & 0.08 & 0.17 & 0.25 & 0.33 \\
    \hline
    Spot & 0.5052\% & 0.5295\% & 0.5500\% & 0.5682\% \\
    \hline
    $Z(0, T_i)$ & 0.9996 & 0.9991 & 0.9986 & 0.9981 \\
    \hline
    Forward $f_i$ & 0.5063\% & 0.5552\% & 0.5927\% & 0.6244\% \\
    \hline
\end{tabular}
\end{center}

\subsection{Custom env: \texttt{vardef}}

For a model $f$ and an instance $x$,
the SHAP value $\phi_i$ for feature $i$ is given by
\begin{equation}
    \label{eq:SHAPValue}
    \phi_i = \sum_{S \subseteq F \setminus \{i\}}
    \frac{|S|! (|F| - |S| - 1)!}{|F|!} \left[
        f_x(S \cup \{i\}) - f_x(S)
    \right]
\end{equation}

\begin{vardef}
    \bitem{$\phi_i(x)$}{
        the SHAP value for feature $i$ for the instance $x$.
    }
    \bitem{$F$}{
        the set of all features
    }
    \bitem{$S$}{
        a subset of all features $F$ excluding $i$
    }
    \bitem{$|S|$}{
        the cardinality of set $S$
    }
    \bitem{$f_x(S)$}{
        prediction of the model $f$ using feature subset $S$
    }
\end{vardef}

The equation \bref{eq:SHAPValue} calculates the average impact
of adding the feature $i$ to all possible subsets of features.

\subsection{Custom env: \texttt{termdef}}

\begin{termdef}
    \bitem{
        Bootstrapping / Stripping
    }{
        This method involves deriving the zero rates by sequentially stripping
        out the coupons, effectively isolating the yield of each cash flow.
        It's done by subtracting the present value of each coupon from the bond price,
        thereby revealing the yield of each cash flow.
    }
    \bitem{
        Parametric / Interpolation Models
    }{
        These models use interpolation techniques combined with smoothing to
        fit the curve data points to a predefined function.
        A common approach in this category is the
        \textit{Nelson-Siegel-Svensson methodology},
        which fits the yield curve data points to a
        specially designed functional form.
    }
\end{termdef}


% ============================================
\section{Refs}

\subsection{\texttt{{\normalfont \textbackslash}bnameref} (clickable refs to sections)}
Here is a ref to the previous section:
\bnameref{sec:SectionToReference}

\subsection{\texttt{{\normalfont \textbackslash}bref} (clickable refs to labels)}

% no real good way to escape backslashes, so use `\textbackslash`.
% Latex then complains about the font, so use \normalfont
\code{{\normalfont \textbackslash}bref}
works for any
\texttt{{\normalfont \textbackslash}label}

\begin{itemize}
    \item ref to an equation: \bref{eq:MyEquation}
    \item ref to a listing: \bref{lst:CodeBlock1}
\end{itemize}

\subsection{Bib refs}

Here is a ref from the \code{.bib} file: \textcite{Guilhoto2018}.

\subsection{Acronyms}

\newacronym{EMH}{EMH}{efficient market hypothesis}
\newacronym{RNN}{RNN}{recurrent neural network}
\newacronym{OCR}{OCR}{optical character recognition}

\begin{itemize}
    \item The first time an acronym is used: \gls{EMH}
    \item The second time an acronym is used: \gls{EMH}
    \item First time with capital letter: \Gls{OCR}
    \item Custom command for title casing: \GlsTC{RNN}
\end{itemize}

% ============================================
\section{Code}

\subsection{Inline Code}

Here is an example of inline code:
\code{df.map(np.log).diff().mean()}

\subsection{Block Code}

\begin{minipage}{\linewidth}
\begin{lstlisting}[
    language=Python,
    frame=single, % border around the code block
    caption=Calculate YTM for a bond in a given IR environment,
    label={lst:CodeBlock1},
]
def calculate_ytm(
    spot_rates: dict[int, float], bond_coupon: float, bond_par: float
) -> float:
    """Calculate YTM for a bond in a given IR environment"""
    _s = "string example"
    coupons_present_value_total = (
        bond_par
        * bond_coupon
        * sum(
            (1 + spot_rate) ** -maturity for maturity, spot_rate in spot_rates.items()
        )
    )
    bond_maturity = max(spot_rates)  # assume spot rates cover maturity of bond
    par_present_value = bond_par * (1 + spot_rates[bond_maturity]) ** -bond_maturity
    coefficients = (
        [coupons_present_value_total + par_present_value]
        + [-bond_coupon * bond_par] * (len(spot_rates) - 1)
        + [-bond_par * (1 + bond_coupon)]
    )
    return np.real(np.roots(coefficients)[0] - 1)
\end{lstlisting}
\end{minipage}
