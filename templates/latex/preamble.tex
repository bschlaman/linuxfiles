% README:
% This preamble should be copy-pasted at the top of all tex documents.
% All choices below MUST be well documented!

% \documentclass[titlepage] ensures titlepage
% (either via `titlepage` env or \maketitle)
% is on its own page
\documentclass[titlepage]{article}

% ==========================================
% SECTION 1.1) PACKAGES: document dimensions
% ==========================================
% set dimensions of the document.
% margin=1in with the rest default is my sweet spot
\usepackage[margin=1in]{geometry}
% uncomment this package for debugging the document geometry
% \usepackage{showframe}
% can be used to configure par spacing; I just use it with defaults
% to remove indents in the document, which is better for mathy docs
\usepackage{parskip}
% provides \thetitle, \theauthor, \thedate
% for using in the `titlepage` env
\usepackage{titling}
% nice headers.  What populates \leftmark and \rightmark
% is determined by the document type; below is a simple config I like
\usepackage{fancyhdr}
\pagestyle{fancy}
% \slshape to slant instead of italics with \itshape
\lhead{\slshape\nouppercase\leftmark}
\rhead{\theauthor}

% ============================================
% SECTION 1.2) PACKAGES: fonts and typeography
% ============================================
\usepackage{amsmath} % provides many commands like \text
\usepackage{amssymb} % provides \triangleq
\usepackage{bbm} % provides \mathbbm
\usepackage{mathrsfs} % provides \mathscr

% =========================================
% SECTION 1.3) PACKAGES: tables and figures
% =========================================
% tabularx must be imported after booktabs!
\usepackage{booktabs} % provides \toprule, \midrule, \bottomrule
\usepackage{tabularx} % used for auto-width columns in some custom envs
\usepackage{float} % floating tables and figures
\usepackage{graphicx} % provides \includegraphics for images
% provides additional configuration for graphics like framing
% `export` exports these options to \includegraphics;
% i.e. \includegraphics[frame]
\usepackage[export]{adjustbox}

% =============================
% SECTION 1.4) PACKAGES: errata
% =============================
\usepackage{framed} % nice framed sections
\usepackage{mdframed} % needed for faside
% Used for placeholder text.
\usepackage{lipsum}
% `lipsum` will cause compilation warning unless the
% following hyphen substitute is explicitly declared
\usepackage{hyphsubst} % silence lipsum warning
\HyphSubstLet{latin}{english} % silence lipsum warning

% ===========================================
% SECTION 1.5) PACKAGES: color and highlights
% ===========================================
% provides \hyperref and \href; clickable links within doc and externally
\usepackage[
    colorlinks=true, % color links instead of boxes
    allcolors=blue, % color of link
    psdextra, % render some unicode characters in pdf bookmarks
]{hyperref}
% Use `\( \)` in section headers (instead of `$`) along with option `psdextra`
% to avoid hyperref complaints about "math shift" being removed from pdf bookmakrs.
% While hacky, this does preserve syntax highlighting unlike something like
% \newcommand\pdfmath[1]{\texorpdfstring{$#1$}{#1}}
\pdfstringdefDisableCommands{
    \def\({}
    \def\){}
}
\usepackage[table,dvipsnames]{xcolor} % dvipsnames for extra colors like RedViolet
\usepackage{soul} % for \hl
\usepackage{listings} % code highlights

% ==================================================
% SECTION 1.5) PACKAGES: bibliography and glossaries
% ==================================================
% glossaries must be loaded after hyperref
% automake is required to automatically compile required files
\usepackage[automake,acronym]{glossaries}
% there is no out-of-the-box title case solution, so defining a custom one
% this does not carry over to the Acronyms glossary section
\newcommand{\GlsTC}[1]{\glslink{#1}{\glsentrytitlecase{#1}{long}}}
\makeglossaries
\usepackage{biblatex}
% bib resource for academic-style papers and books
\addbibresource{refs.bib}
% bib resource for links
\addbibresource{links.bib}
% cite urls with a title and hyperlink to the full url in the bibliography
\newcommand{\urlcite}[1]{\citetitle{#1} \cite{#1}}


% =============================
% SECTION 2.1) CUSTOM: commands
% =============================

\newcommand{\node}[2]{\text{Node}(#1, #2)}
% command for a clickable linked ref in the doc; looks like '(3)'
% this should be used for things OTHER than equations,
% because the behavior is basically the same as \eqref
\newcommand{\bref}[1]{\hyperref[#1]{(\ref*{#1})}}
% command for a clickable named ref in the doc; looks like '(3 Some section)'
\newcommand{\bnameref}[1]{(\ref{#1} \nameref{#1})}
% simple centered rule for visual sectioning.
% some additional spacing below it (3x height of letter 'x')
\newcommand{\brule}{
    \begin{center}
    \rule{0.75\textwidth}{0.4pt}\\[3ex]
    \end{center}
}

% =================================
% SECTION 2.2) CUSTOM: environments
% =================================

% `faside` (fmt-aside) is a grey block used for asides
\newmdenv[
    linecolor=gray,
    linewidth=0.3em,
    topline=false,
    bottomline=false,
    rightline=false,
    backgroundcolor=gray!20,
    leftmargin=1em,
    rightmargin=1em,
    skipabove=2ex,
    skipbelow=2ex,
    nobreak=true,
]{faside}

% `termdef` (term-definition) is a convenience env
% which creates a table of terms and definitions.
% Should be used with `\bitem`
\newenvironment{termdef}{
    \renewcommand{\arraystretch}{1.5}
    \center
    % >{x} inserts `x` before each column entry
    \tabularx{0.8\textwidth}{>{\bfseries}p{0.2\textwidth}|X}
    % Due to tabularx weirdness, I need to use the command
    % versions of things like \tabularx and \center.
    % The `tabular` version of this is simpler / cleaner,
    % but I lose the ability to have dynamically sized columns.
    \toprule
    \textbf{Term} & \textbf{Definition} \\
    \midrule
}{
    \bottomrule
    \endtabularx
    \endcenter
    \renewcommand{\arraystretch}{1.0}
}
% `vardef` (variable-definition) is a convenience env
% which creates a table of variables and their definitions.
% Indtended for use as a "where" clause after equations.
% Should be used with `\bitem`
\newenvironment{vardef}{
    \renewcommand{\arraystretch}{1.5}
    \center
    \rowcolors{2}{CodeInlineBackground}{}
    \tabularx{0.8\textwidth}{l|X}
    \toprule
    \textbf{Variable} & \textbf{Definition} \\
    \midrule
}{
    \bottomrule
    \endtabularx
    \endcenter
    \renewcommand{\arraystretch}{1.0}
}
% Weird thing required for tabularx to handle rowcolors offset correctly.
% Just accept that it works and is required.
\newcounter{tblerows}
\expandafter\let\csname c@tblerows\endcsname\rownum
% `\bitem` (b-item) encapsulates a 2 column table row
\newcommand{\bitem}[2]{
    #1 & #2 \\
}

% ===========================================
% SECTION 3.1) CODE: color defs
% ===========================================
% color definitions.  note \definecolor: color model `gray` - 0 to 1
\definecolor{CodeBlockBackground}{rgb}{0.96,0.96,0.93}
\definecolor{CodeInlineBackground}{gray}{0.93}
% ===========================================
% SECTION 3.2) CODE: inline code
% ===========================================
\sethlcolor{CodeInlineBackground}
\newcommand{\code}[1]{\texttt{\textcolor{RedViolet}{\hl{\;#1\;}}}}
% ===========================================
% SECTION 3.3) CODE: block code
% ===========================================
% configure the lstlistings environment
\lstset{
    backgroundcolor=\color{CodeBlockBackground},
    commentstyle=\color{Gray},
    keywordstyle=\color{purple},
    numberstyle=\tiny\color{Gray},
    stringstyle=\color{OliveGreen},
    basicstyle=\ttfamily\footnotesize,
    breakatwhitespace=false,
    breaklines=true,
    captionpos=b,
    keepspaces=true,
    numbers=left,
    numbersep=5pt,
    showspaces=false,
    showstringspaces=false,
    showtabs=false,
    tabsize=2
}

% ================================================
% SECTION 4) Declare title, author, date, subtitle
% ================================================
% `titling` does not have builtin for subtitle, so create it
\newcommand{\subtitle}[1]{\def\thesubtitle{#1}}
% surrounding functions with no args (like \LaTeX below)
% in brackets improves spacing around them
\title{Brendan's {\LaTeX} Template}
\subtitle{
    Preamble, titlepage, and examples for use in my math-focused {\LaTeX} documents.
}
\author{Brendan Schlaman}
\date{\today}


% =====================================================
% SECTION 5) CUSTOM math operators and common equations
% =====================================================
% these are my operators and notation for
% some common concepts in math and finance
\DeclareMathOperator{\Cov}{Cov}
\DeclareMathOperator{\Corr}{Corr}
\DeclareMathOperator{\Var}{Var}
\DeclareMathOperator{\Bias}{Bias}
\DeclareMathOperator{\PV}{PV}
\DeclareMathOperator{\VaR}{VaR}
\DeclareMathOperator{\ES}{ES}
\DeclareMathOperator*{\argmin}{argmin}
\DeclareMathOperator*{\argmax}{argmax}
\newcommand{\CC}{\mathbb{C}}
\newcommand{\RR}{\mathbb{R}}
\newcommand{\EE}{\mathbb{E}}
\newcommand{\VV}{\mathbb{V}}
\newcommand{\PP}{\mathbb{P}}
\newcommand{\QQ}{\mathbb{Q}}
\newcommand{\TT}{\mathbb{T}}
\newcommand{\N}{\mathcal{N}}
\newcommand{\F}{\mathcal{F}}
\newcommand{\R}{\mathcal{R}}
\newcommand{\E}{\mathcal{E}}
\newcommand{\B}{\mathcal{B}}
\newcommand{\X}{\mathcal{X}}
\newcommand{\G}{\mathcal{G}}
\newcommand{\D}{\mathcal{D}}
% `\renewcommand` must be used, since these were existing commands in tex.
% I am fine with overriding them, as I don't use their defaults.
\renewcommand{\S}{\mathcal{S}}
\renewcommand{\L}{\mathcal{L}}
\renewcommand{\H}{\mathcal{H}}

% \newcommand{\Ito}[1]{Itô #1 \eqref{eq:Ito#1}}
% \newcommand{\Ito}{Itô}
% TODO: latex messes up the spacing when using \Ito instead of \Ito{}
% followed by a space
\makeatletter
\newcommand{\Ito}[1][]{%
    Itô%
    \ifx\@empty#1\@empty
    \else
        ~#1 \eqref{eq:Ito#1}%
    \fi
}
\makeatother